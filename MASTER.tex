\documentclass{beamer}

\usepackage{subfiles}
\usepackage{framed}
\usepackage{amsmath}
\usepackage{amssymb}


\begin{document}

%=============================================================== %
  
  \begin{figure}
\centering
\includegraphics[width=1.1\linewidth]{pythonlogo}

\end{figure}

  
%=============================================================== %
  
  
 \textbf{What is Python?}
 \begin{quote}
  Python is an interpreted, object-oriented, high-level programming language with dynamic semantics. Its high-level built in data structures, combined with dynamic typing and dynamic binding, make it very attractive for Rapid Application Development, as well as for use as a scripting or glue language to connect existing components together. 
 \end{quote}
 
 (Python.org)
  

%=============================================================== %
  
   
   \textbf{What is Python?}
  \begin{quote}
   Python's simple, easy to learn syntax emphasizes readability and therefore reduces the cost of program maintenance. Python supports modules and packages, which encourages program modularity and code reuse. The Python interpreter and the extensive standard library are available in source or binary form without charge for all major platforms, and can be freely distributed.
  \end{quote}
   (Python.org)
  
%=============================================================== % 
 
   
  \textbf{History of Python}
  \begin{quote}
 Python was created in the early 1990s by Guido van Rossum at Stichting Mathematisch Centrum in the Netherlands as a successor of a language called ABC. Guido remains Python’s principal author, although it includes many contributions from others.
 \end{quote}
 


\subfile{00-leadout.tex}
\subfile{01-Introduction.tex}
\subfile{01-versionsconventions.tex}
\subfile{02-visualization.tex}
\subfile{02-packages.tex}
%===================================================== %
 
 \huge
Three Core Packages
\begin{enumerate}
\item numpy
\item pandas
\item scipy
\end{enumerate}
 
%====================================================== %
\subfile{numpy.tex}
\subfile{pandas}
\subfile{scipy.tex}
\subfile{textmining.tex}
\subfile{03-datastructures.tex}
\subfile{06-specialarrays.tex}

\subfile{09-probdistributions.tex}

\end{document}
