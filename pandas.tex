\documentclass[MASTER.tex]{subfiles}
\begin{document}
 %====================================================%
  
  \frametitle{pandas}
   
  \begin{figure}
   \centering
   \includegraphics[width=0.7\linewidth]{pandaslogo}
  \end{figure}
   
    \textit{pandas} is a high-performance module that provides a comprehensive set of structures for working with
   data. 
    \textit{pandas} excels at handling structured data, such as data sets containing many variables, working with
   missing values and merging across multiple data sets. 
   
  
%==============================================%
 
\begin{figure}
\centering
\includegraphics[width=0.55\linewidth]{pydatabook}

\end{figure}


 

%====================================================%
 
 \frametitle{pandas}
  


  
  While extremely useful, \textit{pandas} is not an essential
 component of the Python scientific stack unlike NumPy, SciPy or matplotlib, and so while pandas doesn’t
 make data analysis possible in Python, it makes it much easier.  textit{pandas} also provides high-performance,
 robust methods for importing from and exporting to a wide range of formats.
  
 
%====================================================%
 
 \frametitle{pandas}
 
\noindent \textbf{Data Structures}\\

\textit{pandas} provides a set of data structures which include \texttt{Series}, \texttt{DataFrames} and \texttt{Panels}.
 
\item \textbf{Series} are 1-dimensional
 arrays.
  \textbf{DataFrames} are collections of Series and so are 2-dimensional, 
\item  \textbf{Panels} are collections ofDataFrames,
 and so are 3-dimensional.
  
 

\end{document}
