\documentclass[MASTER.tex]{subfiles} 
\begin{document} 
	
	
\begin{frame}
	\huge
\[ \mbox{Other Interesting Python Packages} \]
\end{frame}




\begin{frame}
	\begin{figure}
\centering
\includegraphics[width=0.7\linewidth]{SKL-logo}

\end{figure}

\end{frame}
%===========================================================%

\begin{frame}
	\frametitle{scikit.learn}
		\begin{figure}
			\centering
			\includegraphics[width=0.5\linewidth]{SKL-logo2}
		
		\end{figure}
		
	\begin{itemize}
	\item scikit-learn is an open source machine learning library for the Python programming language. 
	\item scikit-learn features various classification, regression and clustering algorithms including support vector machines, logistic regression, naive Bayes, random forests, gradient boosting, k-means and DBSCAN. \item scikit-learn is designed to interoperate with the Python numerical and scientific libraries NumPy and SciPy.
	\end{itemize}
\end{frame}
%===========================================================%
\begin{frame}
\textbf{Sci-Kit Learn Site info}
	\begin{figure}
\centering
\includegraphics[width=1.1\linewidth]{SKLsiteinfo}
\end{figure}
\end{frame}
%===========================================================%
\begin{frame}
\begin{figure}
\centering
\includegraphics[width=0.9\linewidth]{SKLCheatSheet}

\end{figure}
\end{frame}
%===========================================================%
\begin{frame}
	\begin{figure}
		\centering
		\includegraphics[width=0.9\linewidth]{SKLCheatSheet2}
		
	\end{figure}
\end{frame}
%=========================================================== %
\begin{frame}
	\begin{figure}
\centering
\includegraphics[width=1.1\linewidth]{machinelearningquotes}
\end{figure}
\Large Machine Learning is statistics minus any checking of models or assumptions
\end{frame}
%============================%
\begin{frame}
	\frametitle{The Data Science Profession}
	Data Science Retreat (Berlin)
	\begin{quote}
		MOOC have not  decreased the barrier of entry to machine-learning.
		
		
		Nowadays, you cannot be 'the guy who knows how to run (insert off-the-shelf-algo-here)'. 
		
		
		In dataland, that's the equivalent to being a code monkey. MOOCs and superb libraries (scikit-learn, R's ecosystem) made 
		sure there is plenty of people who can throw say a random forest to a problem. In the modern world, this is not adding that much value. 
	\end{quote}
\end{frame}
%===========================================================%
\begin{frame}
\frametitle{Other Packages}
\large
\textbf{pytz and babel}\\
ptyz and babel provide extended support for time zones and formatting information.\\ \bigskip
\textbf{rpy2 }\\
rpy2 provides an interface for calling R 3.0.x in Python, as well as facilities for easily moving data between
the two platforms.\\ \bigskip
\textbf{PyTables and h5py }\\
PyTables and h5py both provide access to HDF5 files, a flexible data storage format optimized for numeric
data.
\end{frame}
\end{document}